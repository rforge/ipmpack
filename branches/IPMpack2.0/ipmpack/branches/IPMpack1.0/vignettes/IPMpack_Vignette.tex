\documentclass{article}

\usepackage{amsmath}
\usepackage{amscd}
\usepackage[tableposition=top]{caption}
\usepackage{ifthen}
\usepackage[utf8]{inputenc}
\usepackage{longtable,pdflscape} 
\usepackage{mathptmx}
\usepackage{hyperref}
\usepackage{Sweave}
\usepackage{tikz}
\usepackage{pgf}
\usepackage{a4wide}


 %\SweaveOpts{prefix.string=Guide/} 
 
\begin{document}

\title{IPMpack: an R package for demographic modeling with Integral Projection Models (v.1.0)}
\author{Jessica Metcalf, Sean M. McMahon, Rob Salguero-Gomez, Eelke Jongejans}
\maketitle


%\section*{IPMpack: an R package for demographic modeling}
%\makeabstract
The goal of IPMpack is to provide a suite of demographic tools based
on Integral Projection Models (IPMs) to support biologists interested in
making projections for populations where demography is strongly linked to a continuous variable, such as size. The package includes functions that can take data, such as size or age, as well as environmental covariates, and build models of growth, survival and fecundity. Functions are defined that then take these
statistical models and construct IPMs. IPMpack has tools that compare different functional forms for the underlying statistical models, plotting them and returning AIC scores, as well as tools for diagnostic tests of the IPM models themselves. There are also methods to build population models for varying environments, use Bayesian methods to sample population parameters,  estimate longevity and passage time, sensitivity and elasticity (of either parameters or matrix elements), and much more.

This vignette is intended to introduce the biologists with a wide range of quantitative skills to the concepts of IPMs as well as the implementation of IPMpack.  This vignette is for IPMpack version $1.0$, and so we encourage users to contact the IPMpack team at \href{IPMpackTeam@gmail.com}{IPMpackTeam@gmail.com} with any feedback or mistakes they find.  We also host a blog at R-forge \href{http://ipmpack.r-forge.r-project.org/}{IPMpack Web Site} that contains news of updates, new features, and announcements of papers and meetings relevant to IPMs. 
 
\newpage

\section{Introduction to Integral Projection Models}
An Integral Projection Model (IPM) is a demographic tool to explore the dynamics of populations where individuals' fates depend on state variables that are continuous (e.g., weight, diameter at breast height, height, limb length, rosette diameter) or quasi-continuous (e.g., number of leaves, age, number of reproductive structures) and may be a mixture of discrete and continuous. IPMs track the distribution of individuals $n$ across these state variables between census times (e.g., year $t$ and year $t+1$) by projecting from models that define the underlying vital rates (e.g., survival, growth, and reproduction) as a function of the (quasi-)continuous state variables. For detailed introductions to IPMs, see Easterling et al. ($2000$), and Ellner \& Rees ($2006$, $2007$). 

Briefly, an IPM is defined by a kernel $K$ that represents probabilities of growth between discrete or continuous stages, survival across these stages, and the production of offspring and offspring recruitment.   For example, in the simplest case, where the population is structured by a continuous covariate, size, then 
\begin{equation}
n(y, t+1) = \int\limits_{L}^{U} K(y, x) n(x, t) \, dx       
\end{equation}
where $n(y, t+1)$ is the distribution across size $y$ of both established and new individuals in census time $t+1$, $n(x, t)$ the distribution across size of individuals in census time $t$, and $L$ and $U$ the lower and upper size limits modeled in the IPM, respectively. 

 Multiple functional forms for both demographic processes as well as their error structures can be easily accommodated with IPMpack. The $F$ kernel (equation 4) describes per-capita contributions of reproductive individuals to number of new individuals at the next census. Multiple size-dependent or size-independent vital rates can be fitted within the $F$ kernel, reflecting for example reproductive probability, number of reproductive structures (e.g. flowers in plants, basidia in fungi), number of propagules within reproductive structure (e.g. seeds for plants), and so on. Additionally, a range of constants ($c_1$, $c_2$, ...) can be included if there are no data for a stage.  Finally, the $F$ kernel definition includes a probability density function describing the size of offspring recruiting into the population, $f_d$, 
\begin{equation}
n(y, t+1) = \int\limits_{L}^{U} K(y, x) n(x, t) dx = \int\limits_{L}^{U}
[T(y, ) + F(y, x)] n(x, t) dx
\end{equation}

\begin{equation}
n(y, t+1) = \int\limits_{L}^{U} T(y, x) n(x, t) dx = \int\limits_{L}^{U}surv(x)growth(y, x)dx    
\end{equation}

\begin{equation}
n(y, t+1) = \int\limits_{L}^{U} F(y, x) n(x, t) dx = \int\limits_{L}^{U}
c_1 c_2 c_3 ... fec1(x)fec2(x)fec3(x)...f_d(y, x)dx     
\end{equation}
After numerically solving these kernels, key ecological and evolutionary quantities such as the population rate of increase $\lambda$, the stable population size structure, the net reproductive rate $R_0$, and many others can be estimated (see Caswell $2001$ for more a comprehensive discussion). 

Essentially, the same tools are available for IPMs as for discrete projection matrices (matrix population models), e.g., estimation of population growth rate, sensitivities, elasticities, life table response experiment [LTRE] analyses, passage time calculations, etc (Caswell $2001$, Cochran \& Ellner $1992$, and others). The main difference between an IPM and a matrix model is that while in discrete projection matrices the number of classes (i.e., number of stages in the life cycle of the study species) must be defined {\tt a priori}, IPMs impose the discretization of the three-dimensional surface defined by equation 1 in the last step. This produces a typically large matrix (e.g., $100$ x $100$ cells) that is more robust to biases from matrix dimensionality (Zuidema et al. $2010$, Salguero-Gomez \& Plotkin $2010$) and sample size (Ramula et al. $2009$) than classical matrix models.  

The goal of IPMpack is to provide a centralized set of quantitative techniques based on IPMs to help ecologists and evolutionary biologists model populations. IPMpack v. 1.0 can accommodate multiple vital rates from complex life cycles all grouped into two main sub-kernels: $T$ and $F$ (equation 2) \footnote{Note than in the seminal paper by Easterling et al. ($2000$) this kernel was referred to as $P$, but here we follow the terminology by Caswell ($2001$) and call it $T$ instead). The $T$ kernel (equation 3) describes {\tt growth} between demographic censuses conditional on individuals' survival ({\tt surv}).
}.

This vignette will now walk through the steps of a basic IPM analysis.  We first describe the kind of data necessary to build an IPM.  If a user begins `from scratch', they must input data in a specific format (described below).  However it is possible to jump past this step and use IPMpack capabilities on IPMs that were developed outside of IPMpack.  That is, if a user wants quick diagnostic routines, figures and summary statistics on an IPM matrix already built, IPMpack can readily accommodate that.   However there are some features that, because of the object-oriented coding require some specific structures (and other features that do not).  Please refer to the manual files and the rest of this vignette for this information.  But however a user wants to implement IPMpack, the vignette will begin at the beginning with data set up.  We will then walk through how to build and analyse a basic IPM model.  More complex models will be introduced later, with options to create unique class objects and methods, as well as run comparative model testing and Bayesian implementations.

\section{Getting started: setting up the data for IPMpack}
For users who prefer to define IPM matrices using their own
statistical tools, there is no requirement for the data to be in any
particular format, and most of the functions in IPMpack will operate
on the matrices directly (e.g., life expectancy, sensitivity of matrix
elements, etc).  However, to use IPMpack's full capacities, the
individual-level demographic data must be organized in a specific
format in R: a {\tt data frame} where each row represents one observation of an organism in the population at one census time $t$ with the following column names:  
\begin{itemize}
\item  {\tt size}: size of individuals in census time $t$  $^*$
\item  {\tt sizeNext}: size of individuals in census time $t+1$  $^*$
\item  {\tt surv}: survival of individuals from census time $t$ to  $t+1$ (contains: 0 for death or 1 for survival) $^*$
\item  {\tt fec1, ...}: as many columns as desired relating size to sexual reproduction. For example, this might be: 
  \begin{itemize}
  \item {\tt fec1}: probability of reproduction (output: 0 for no reproductive or 1 for reproductive)
  \item {\tt fec2}: number of reproductive structures (output: 1, 2, 3, $...$) when individual is reproductive, that is, when fec1 = 1
  \item {\tt fec3}: number of propagules (output: 1, 2, 3, $...$) per reproductive structure (e.g. seeds per flower in reproductive plant individual)
  \item ...
  \end{itemize}
  The default construction for the analytical part of IPMpack is such that any columns for which the column label contains $"fec"$ will be included in the analysis of the reproductive part of the life cycle (kernel $F$) automatically. This default can be over-ridden so that specific columns are identified for IPMpack functions to use.   
\item  {\tt stage}: stage of individuals in census time $t$. For rows in the  data frame where {\tt size} is not an NA, then this must be the word ``continuous''. Where {\tt size} is NA, any variety of named discrete stages  may be defined (e.g. ``seed bank''). If this column is missing, many procedures in IPMpack are designed to simply fill in this column assuming that only ``continuous'' state variables describe the life cycle of the species, i.e. there are no discrete stages. 
\item  {\tt stageNext}: stage of individuals in census time $t+1$;  likewise, this column is not essential for many procedures in  IPMpack.  
\item  {\tt number}: number of individuals corresponding to each row in the data frame. For all rows corresponding to movement between continuous stages, this value will be $1$, but for movement between  discrete stages (e.g., from ``dormant seeds'' to ``seeds ready to  germinate'') then this number may be $>1$, potentially directly  reflecting observed individuals in the data. This information avoids having a data frame with a row for every discrete stage (e.g. seed). As above, many  proceedures in IPMpack will simply assume that this value is always 1. 
\item  {\tt covariate}: value of a discrete covariate in census time  $t$, such as light environment at time $t$, age at $t$, patch at $t$, etc. 
\item  {\tt covariateNext}: value of a discrete covariate in census time $t+1$.
\item  ...any other covariates of interest, named as desired by the user are possible too (e.g., precipitation, habitat, temperature, etc).
\end{itemize}

The $^*$ symbol above indicates the minimum columns in the data frame required to obtain passage time and life expectancy calculations. These values form the $T$ kernel. If sufficient additional columns are available, a full life-cycle model, containing the $F$ kernel, can be produced and further analyses are possible.  Although {\tt size} and {\tt sizeNext} can be transformed, many of the utility functions assume no transformations in columns in the original data frame pertaining to fertility. Transformations can be formally called in various parts of the package and appropriate $F$ matrices built that account for these transformations. In addition, users may also define IPMs independently, and then introduce them into IPMpack for application of further utility functions (sensitivities, stochastic growth rates, etc). 


\section{The basics: building an IPM}

First, the user must load the IPMpack package from cran into R. 

\begin{Schunk}
\begin{Sinput}
> library("IPMpack")